% s\"amtlich Pakete die ben\"otigt werden

% ----- Umlaute -----
\usepackage[utf8]{inputenc}                       % Kodierung der Eingabezeichen
\usepackage[T1]{fontenc}                          % Kodierung der Schriften

% ----- Rechtschreibung -----
\usepackage[ngerman]{babel}                       % neue deutsch Rechtschreibung
\usepackage{setspace}                             % Einstellung f\"ur Zeilenabstand
\usepackage{geometry}                             % Einstellung der Seitenabst\"ande

% ----- Schriftart -----                          
\usepackage{courier}                             
\usepackage{relsize}                              % Schriftgr\"o\ss{}e relativ festlegen

% ----- sonstiges -----
\usepackage{url}                                  % URL-Highlighting
\usepackage[
    automark,                                     % Kapitelangaben in Kopfzeile automatisch erstellen
    headsepline,                                  % Trennlinie unter Kopfzeile
    ilines                                        % Trennlinie linksb\"undig ausrichten
]{scrpage2}
\usepackage{makeidx}
%\usepackage{natbib}
\usepackage{xcolor}

% ---- Abk\"urzungsverzeichnis ----
\usepackage[intoc]{nomencl}
\let\abbrev\nomenclature
\renewcommand{\nomname}{Abk\"urzungsverzeichnis}  % Anzeigenamen \"andern
\setlength{\nomlabelwidth}{.20\hsize}
\renewcommand{\nomlabel}[1]{#1 \dotfill}          % Zwischenraum mit Punkt f\"ullen
\setlength{\nomitemsep}{-\parsep}

% ----- Bilder -----
\usepackage[pdftex]{graphicx}                     % Bildunterst\"utzung
\usepackage{wrapfig}                              % Text um Bilder flie\ss{}en lassen
\graphicspath{{bilder/}}                          % Pfad zu den Bilddatein


% ----- Mathesachen -----
\usepackage{amsmath}
\usepackage{amsfonts}
\usepackage{amssymb}


% ---- PDF-Optionen ----
\usepackage[
    bookmarks,
    bookmarksopen=true,
    colorlinks=true,    
% +++ farbige Links +++
    linkcolor=red,                                 % einfache interne Verk"un\pfungen
    anchorcolor=black,                             % Ankertext
    citecolor=blue,                                % Verweise auf Literaturverzeichniseintr\"age im Text
    filecolor=magenta,                             % Verkn\"upfungen, die lokale Dateien \"offnen
    menucolor=red,                                 % Acrobat-Men\"upunkte
    urlcolor=cyan, 
% +++ schwarze Links +++
    %linkcolor=black,                               % einfache interne Verknüpfungen
    %anchorcolor=black,                             % Ankertext
    %citecolor=black,                               % Verweise auf Literaturverzeichniseinträge im Text
    %filecolor=black,                               % Verknüpfungen, die lokale Dateien öffnen
    %menucolor=black,                               % Acrobat-Menüpunkte
    %urlcolor=black, 
    backref,
    plainpages=false,                              % zur korrekten Erstellung der Bookmarks
    pdfpagelabels,                                 % zur korrekten Erstellung der Bookmarks
    hypertexnames=false,                           % zur korrekten Erstellung der Bookmarks
    linktocpage                                    % Seitenzahlen anstatt Text im Inhaltsverzeichnis verlinken
]{hyperref}

% ---- f\"ur lange Tabellen ----
\usepackage{longtable}
\usepackage{array}
\usepackage{ragged2e}
\usepackage{pdflscape} 
\usepackage{lscape}

% ---- color ----
\usepackage{colortbl}                              % zum einf�rben von Tabellenspalten
\usepackage{xcolor}                                % f�r eigene Farbdefinitionen

\usepackage{tikz-uml}
\usepackage{tikz}
\usepackage{ifthen}
\usepackage{xstring}
\usepackage{calc}
\usepackage{pgfopts}

\usetikzlibrary{shapes,arrows}